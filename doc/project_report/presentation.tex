\documentclass{beamer}
\usepackage{graphicx}
\begin{document}
\title{The Gate Opener: A CS251 Project by Group 24}   
\author{Gaurav Jain\\140020104\newline \newline Vishal Meena \\140050010 \newline \newline Ankur Pooniya \\ 140050016}


\date{\today} 

\frame{\titlepage} 

\frame{\frametitle{Introduction}
\begin{itemize}
\item This project is a simulation of the Rube-Goldberg Machine, created using Box2D a physics engine of C++. It is called 'The Gate Opener'. The basic theme is that a series of changes take place just in time to open the gate for the vehicle to pass. \pause 

\item The machine implements spring-mass system, Newton's pendulum, domino effect and pulley systems. The project report contains an elaborate description of the individual parts and other project specifications. \pause

\item Another interesting thing about the simulation is that we can use ideal cases, even superideal cases. This opens up vast possibilities in the simulation.
\end{itemize} 
}

\frame{\frametitle{Why this Project?}
\begin{itemize}
\item The essence of Rube-Goldberg Machine is that it accomplishes a very trivial task via a very complicated route. \pause 
\item In our project, the car moves at a steady rate while a series of changes take place on top. The gate opens just in time for the vehicle to pass through, thus completing the purpose of the machine. \pause 
\item The project is not so much of utility in terms of practical purposes, but it is great learning experience. 
\end{itemize} 
}

\frame{\frametitle{Challenges we Faced}
\begin{itemize}
\item The most difficult part of the project was getting the precision right for all processes going on simultaneously.  \pause 
\item Most processes in the machine act just on the edge. So changing even a single component requires changes in every related component.\pause 
\item The tradeoffs between friction, restitution, density and damping constants need to be carefully analysed and chosen. \pause
\item One weakness which still persists is that the simulation is not triggered by the car.
\end{itemize} 
}

\frame{\frametitle{Efforts}
\begin{itemize}
\item The project required efforts from all three team members. Creativity has to be accompanied with labour to produce results.\pause
\item Coding can be exasperating at times. It requires patience. All the team members were there for each other in times of need.\pause 
\item Coordination between team members is the crux of any team project. Git repositories allow version control and sync seamlessly. \pause
\end{itemize} 
}


\frame{\frametitle{Acknowledgements}
\begin{itemize}
\item Needless to say, we would like to thank Prof. Sharat Chandran for his guidance and motivation during labs and classes. \pause 
\item We were also helped by our TA, Niharika Kurade. \pause
\item We are very grateful to everyone who contributed on Piazza to solve queries regarding the project.
\end{itemize} 
}

\end{document}
